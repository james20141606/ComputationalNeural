%\documentclass[openany]{book}
\documentclass[11pt]{article}
\usepackage{ctex}
\usepackage{indentfirst}
\usepackage[toc,page]{appendix}
%\usepackage[titletoc,title]{appendix}
\usepackage{subfig}
\DeclareMathAlphabet{\pazocal}{OMS}{zplm}{m}{n} %for fancy L
\usepackage{epsfig, float,array,tabu,longtable,}
%\usepackage{hyperref,wrapfig}
\usepackage{enumerate}
\usepackage[at]{easylist}
\usepackage{graphicx,psfrag}
\usepackage{cite}
\usepackage{sectsty}
\usepackage{enumitem}
\usepackage{subfig}
\usepackage{caption}
\newlist{inparaenum}{enumerate}{2}% allow two levels of nesting in an enumerate-like environment
\setlist[inparaenum]{nosep}% compact spacing for all nesting levels
\setlist[inparaenum,1]{label=\bfseries\arabic*.}% labels for top level
\setlist[inparaenum,2]{label=\arabic{inparaenumi}\emph{\alph*})}% labels for second level

\usepackage{epstopdf}
\usepackage{amsmath,esint, setspace, fancyhdr, amsfonts,  blindtext}
\usepackage[normalem]{ulem}

\usepackage{tikz}
\usepackage{rotating}
\usepackage[americanvoltages,fulldiodes,siunitx]{circuitikz}
\usepackage{stackengine}
\usetikzlibrary{matrix}
\usepackage{multirow}

\usetikzlibrary{shapes,backgrounds,patterns}
\usetikzlibrary{mindmap,trees,decorations.markings}
\usetikzlibrary{quotes,angles}
\usepackage{verbatim}
\renewcommand{\baselinestretch}{1}
\setlength{\textheight}{8in}
\setlength{\textwidth}{6.5in}
\setlength{\headheight}{0in}
\setlength{\headsep}{0.3in}
\setlength{\topmargin}{0in}
\setlength{\oddsidemargin}{0in}
\setlength{\evensidemargin}{0in}
\setlength{\parindent}{.2in}
\usepackage{algorithmic}
\usepackage{algorithm,float}
\usepackage{caption}

\makeatletter
\newenvironment{breakablealgorithm}
  {% \begin{breakablealgorithm}
   \begin{center}
     \refstepcounter{algorithm}% New algorithm
     \hrule height.8pt depth0pt \kern2pt% \@fs@pre for \@fs@ruled 画线
     \renewcommand{\caption}[2][\relax]{% Make a new \caption
       {\raggedright\textbf{\ALG@name~\thealgorithm} ##2\par}%
       \ifx\relax##1\relax % #1 is \relax
         \addcontentsline{loa}{algorithm}{\protect\numberline{\thealgorithm}##2}%
       \else % #1 is not \relax
         \addcontentsline{loa}{algorithm}{\protect\numberline{\thealgorithm}##1}%
       \fi
       \kern2pt\hrule\kern2pt
     }
  }{% \end{breakablealgorithm}
     \kern2pt\hrule\relax% \@fs@post for \@fs@ruled 画线
   \end{center}
  }

\begin{document}


\begin{center}
\newcommand{\HRule}{\rule{\linewidth}{0.5mm}} % Defines a new command for the horizontal lines, change thickness here

\textsc{\LARGE Homework 1}  \\[0.5cm] % Name of your university/college
\HRule \\[1cm]
{ \huge  \textbf{神经影像与神经记录方法}}\\[0.5cm] % Title of your document
\HRule \\[1cm]
% If you don't want a supervisor, uncomment the two lines below and remove the section above
\begin{center}
\Large \emph{}\\
\text{生51 陈旭鹏}\\
\text{学号: 2014012882}\\ 
\end{center}
\begin{minipage}{0.4\textwidth}
\begin{center}
{\large \today}
\end{center}
\end{minipage}\\[2cm]
 % Date, change the \today to a set date if you want to be precise
\end{center}


\section{MRI 大脑结构分析}
\subsection{大脑占比}
利用网站http://surfer.nmr.mgh.harvard.edu/fswiki/CorticalParcellation的相关注释,找到数据中每个voxel的编号所对应的脑区,可以观察到大脑皮层的脑区数据编号为1000~2000,且左脑为1000开头,右脑为2000开头。在对应时也把Cingulate考虑了进去。将四个脑区的左右脑对应的voxel分别重新赋值为1~8方便计数,可视化结果如下。

\begin{figure}[h]
\centering
\includegraphics[width = 15cm ]{0.png}
\caption{可视化部分层重新编号的四个脑区的图像}
\end{figure}





\end{document}

